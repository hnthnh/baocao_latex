	%------------------------------------------------------------
	%|					CHAPTER 1		                     	|  
	%| GIỚI THIỆU CHUNG ĐƠN VỊ THỰC TẬP              			|
	%------------------------------------------------------------
	\begin{center}
		\pagenumbering{arabic}\setcounter{page}{2}
	\end{center}
	
	\begin{flushleft}
		
		\fontsize{16}{20}\selectfont
		\section*{CHƯƠNG 1: GIỚI THIỆU CHUNG ĐƠN VỊ THỰC TẬP }
		\addcontentsline{toc}{section}{CHƯƠNG 1: GIỚI THIỆU CHUNG ĐƠN VỊ THỰC TẬP }
		\fontsize{13}{13}\selectfont
        \setcounter{section}{1}
        \subsection{Thông tin về đơn vị thực tập}
		\paragraph{}
        Hệ thống được thiết kế cho các khách sạn vừa và nhỏ nên chỉ tập trung khai thác và xử lý các vấn đề như tạo hóa đơn, chỉnh sửa thông tin hóa đơn, quản lý KH, quản lý nhân viên, v.v . Những công việc trên hoàn toàn có thể làm thủ công. Tuy vậy, việc đó dễ gây ra sao sót ảnh hưởng tới doanh thu của khách sạn và một số vấn đề liên quan. Việc xây dựng hệ thống quản lý khách sạn tạo ra một form mẫu cho các công viêc lặp đi lặp, từ đó người sử dụng dễ dàng tiếp cận làm quen nhanh chóng và người được phân quyền cao dễ dàng quản lý.Phạm vi: Ứng dụng cho các mô hình khách sạn vừa và nhỏ.
	\pagebreak
		\begin{itemize}
			\item[-] Xây dựng hệ thống quản lý khách sạn bao gồm các bước:
			\begin{itemize}
				\item[+]Khảo sát nhu cầu người dùng.
				\item[+]Phân tích hệ thống.
				\item[+]Mô hình hóa hệ thống bằng các sơ đồ cụ thể.
				\item[+]Thiết kế các chức năng của hệ thống.
			\end{itemize}
		\end{itemize}	
			\begin{itemize}
			\item[-]Hệ thống gồm có các chức năng sau:
			\begin{itemize}
				\item[+]Quản lý phòng và các dịch vụ tương ứng.
				\item[+]Quản lý khách hàng.
				\item[+]Lập hóa đơn.
				\item[+]Thanh toán hóa đơn.
				\item[+]Thống kê doanh thu định kì.
				\item[+]Quản lý nhân sự.
			\end{itemize}
		\end{itemize}	