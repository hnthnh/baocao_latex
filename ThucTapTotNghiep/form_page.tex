 %------------------------------------------------------------
	%|							PAGE ii							|
	%------------------------------------------------------------
	\begin{center}
		\setstretch{1.0}
		\fontsize{16}{20}\selectfont
		\textbf{ĐỒ ÁN ĐƯỢC HOÀN THÀNH}\\
		\textbf{TẠI TRƯỜNG ĐẠI HỌC TÔN ĐỨC THẮNG\\} 
	\end{center}
	\setstretch{1.5}
	\fontsize{13}{13}\selectfont
	\paragraph{}
	Tôi xin cam đoan đây là sản phẩm đồ án của riêng chúng tôi và được sự hướng dẫn của thầy Dương Hữu Phúc;. Các nội dung nghiên cứu, kết quả trong đề tài này là trung thực và chưa công bố dưới bất kỳ hình thức nào trước đây. Những số liệu trong các bảng biểu phục vụ cho việc phân tích, nhận xét, đánh giá được chính tác giả thu thập từ các nguồn khác nhau có ghi rõ trong phần tài liệu tham khảo.
	\paragraph{}
	Ngoài ra, trong đồ án còn sử dụng một số nhận xét, đánh giá cũng như số liệu của các tác giả khác, cơ quan tổ chức khác đều có trích dẫn và chú thích nguồn gốc.
	\paragraph{}
	\textbf{Nếu phát hiện có bất kỳ sự gian lận nào chúng tôi xin hoàn toàn chịu trách nhiệm về nội dung đồ án của mình.} Trường đại học Tôn Đức Thắng không liên quan đến những vi phạm tác quyền, bản quyền do tôi gây ra trong quá trình thực hiện (nếu có).
	\begin{flushright}
		TP. Hồ Chí Minh,  ngày   tháng   năm   \\ 
		\textit{Tác giả\\
			(ký tên và ghi rõ họ tên)\\
			\vspace{1.5cm}
			Lương Gia Hân\\
			\vspace{1.5cm}
			Trần Gia Thái\\
			\vspace{1.5cm}}
	\end{flushright}
	\pagebreak
	
	%------------------------------------------------------------
	%|						PAGE iii							|
	%------------------------------------------------------------
	\begin{center}
		\setstretch{1.0}
		\fontsize{16}{20}\selectfont
		\textbf{PHẦN NHẬN XÉT VÀ ĐÁNH GIÁ CỦA GIẢNG VIÊN}\\
	\end{center}
	\setstretch{1.5}
	\fontsize{13}{13}\selectfont
	\textbf{Phần xác nhận của GV hướng dẫn}\\
	................................................................................................................................\\
	................................................................................................................................\\
	................................................................................................................................\\
	................................................................................................................................\\
	................................................................................................................................\\
	\begin{flushright}
		TP. Hồ Chí Minh,  ngày   tháng   năm   \\ 
		(ký tên và ghi rõ họ tên)\\
		\vspace{3cm}
	\end{flushright}
	\setstretch{1.5}
	\fontsize{13}{13}\selectfont
	\textbf{Phần đánh giá của GV chấm bài}\\
	................................................................................................................................\\
	................................................................................................................................\\
	................................................................................................................................\\
	................................................................................................................................\\
	................................................................................................................................\\
	\begin{flushright}
		TP. Hồ Chí Minh,  ngày   tháng   năm   \\ 
		(ký tên và ghi rõ họ tên)\\
		\vspace{3cm}
	\end{flushright}
	\pagebreak
	
	%------------------------------------------------------------
	%|							PAGE iv							|
	%------------------------------------------------------------
	\begin{center}
		\fontsize{16}{20}\selectfont
		\textbf{TÓM TẮT\\} 
	\end{center}
	\setstretch{1.5}
	\fontsize{13}{13}\selectfont
	\paragraph{}
	Trong thời buổi hiện nay, CNTT không chỉ là phương tiện hỗ  trợ cho cuộc sống, nhu cầu cá nhân mà còn được xem như thước đo về sự phát triển của bất kì một tổ chức, quốc gia nào. Việc triển khai các hệ thống quản lý giúp gia tăng năng suất hoạt động một cách hiệu quả, giảm thiểu  tối đa tình trạng sai sót do ảnh hưởng về mặt tâm sinh lý con người. Tuy nhiên, hiện vẫn không ít các khách sạn vừa và nhỏ chỉ thực hiện công việc quản lý khách sạn thủ công. Do đó, đồ án này được thiết kế nhằm tối ưu hóa các công việc thủ công lặp đi lặp lại, tránh tình trạng sai sót và thiếu chính xác trong vấn đề quản lý khách sạn.
	
	\pagebreak
	